\documentclass{article}
\usepackage{graphicx} % Required for inserting images

\title{Thermoacoustic heat pumps}
\author{Thomas Meylaers}
\date{April 2023}

\newcommand{\newpara}
    {
      \bigbreak{}
      \noindent
    }

\begin{document}

\maketitle

\newpage

\begin{abstract}

\end{abstract}

\newpage

\tableofcontents

\newpage


\section{Introduction}
Thermoacoustic heat pumps (TAHP) are a new and upcoming technology. This technology offers an environment-friendly and efficient solution for heating applications.
\newpara{}
In contrast to traditional heating and cooling applications which rely on the vapor compression cycles, TAHP uses the energy stored in sound waves to transfer heat between temperature reservoirs. These mechanisms have a relatively simple design compared to their more-traditional counterparts. TAHP also use more environment-friendly working media such as helium and hydrogen.
Working on these principles, this technology has the potential to offer significant advantages over traditional heating and cooling methods, including reduced energy consumption, lower maintenance costs, and reduced environmental impact.
\newpara{}
This paper will first discuss the physics behind these systems in section X. Section X discusses the applications of TAHP with the difference between standing wave and traveling wave principles in mind. Subsequently, section X discusses the challenges and possible shortcomings of TAHP.\@ This paper will conclude in section X with final remarks.

\section{Working principles}
When a sound wave travels through a tube, gas parcels alternatively compress and expand. This process happens adiabatically. These compressions and expansions change the temperature and pressure of these gas parcels. When the pressure reaches maximum so does the temperature and vice versa. Normally, these temperature differences cancel each other out and the average temperature of the medium stays constant. % FIGURE
A TAHP, however, harnesses these temperature differences in a wave. They do this by using a device called the \emph{stack}. % FIGURE
\newpara{}
The \emph{stack} is designed in such a way that there is a maximum amount of heat transfer from the gas to the stack, but within the stack the heat conduction is limited. The stack should also minimize its obstruction of the gas. A temperature gradient develops on the \emph{stack} when it is placed at a certain location. A heat flow is then generated by placing heat exchangers on both sides of the stack. The reverse of this principle is also true, by applying a temperature gradient to the stack a sound wave can be produced. This system is called a thermoacoustic heat engine or a prime mover, but will not be discussed further in this paper.% FIGURE
\newpara{}
This paper will further discuss the two main types of TAHP, the \emph{standing wave} TAHP and the \emph{traveling wave} TAHP.

\subsection{Standing Waves}
\subsection{Travelling Waves}

\section{Applications}

\section{Future challenges and limitations}

\section{Conclusion}

\end{document}
