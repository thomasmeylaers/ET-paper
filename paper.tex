\documentclass{article}
\usepackage{graphicx} % Required for inserting images

\title{Thermoacoustic heat pumps}
\author{Thomas Meylaers}
\date{April 2023}

\newcommand{\newpara}
    {
      \bigbreak{}
      \noindent
    }

\begin{document}

\maketitle

\newpage

\begin{abstract}

\end{abstract}

\newpage

\tableofcontents

\newpage


\section{Introduction}
Thermoacoustic heat pumps (TAHP) are a new and upcoming technology. This technology offers an environment friendly and efficient solution for heating applications.
\newpara{}
In contrast to traditional heating and cooling applications which rely on the vapor compression cycles, TAHP uses the energy stored in sound waves to tranfser heat between temperature reservoirs. These mechanisms have a relatively simple design compared to their more traditional counterparts. TAHP also use more environment friendly working media such as helium and hydrogen.
Working on these principles, this technology has the potential to offer significant advantages over traditional heating and cooling methods, including reduced energy consumption, lower maintenance costs, and reduced environmental impact.
\newpara{}
This paper will first dicuss the physics behind these systems in section X. Section X discusses the applications of TAHP with the difference of standing wave and travelling wave principles in mind. Subsequently, section X discusses the challenges and possible shortcomings of TAHP.\@ This paper will conclude in section X with final remarks.

\section{Working principles}
\subsection{Standing Waves}
\subsection{Travelling Waves}

\section{Applications}

\section{Future challenges and limitations}

\section{Conclusion}

\end{document}
